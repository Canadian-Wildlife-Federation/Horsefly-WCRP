%% Generated by Sphinx.
\def\sphinxdocclass{jupyterBook}
\documentclass[letterpaper,10pt,english]{jupyterBook}
\ifdefined\pdfpxdimen
   \let\sphinxpxdimen\pdfpxdimen\else\newdimen\sphinxpxdimen
\fi \sphinxpxdimen=.75bp\relax
\ifdefined\pdfimageresolution
    \pdfimageresolution= \numexpr \dimexpr1in\relax/\sphinxpxdimen\relax
\fi
%% let collapsible pdf bookmarks panel have high depth per default
\PassOptionsToPackage{bookmarksdepth=5}{hyperref}
%% turn off hyperref patch of \index as sphinx.xdy xindy module takes care of
%% suitable \hyperpage mark-up, working around hyperref-xindy incompatibility
\PassOptionsToPackage{hyperindex=false}{hyperref}
%% memoir class requires extra handling
\makeatletter\@ifclassloaded{memoir}
{\ifdefined\memhyperindexfalse\memhyperindexfalse\fi}{}\makeatother

\PassOptionsToPackage{warn}{textcomp}

\catcode`^^^^00a0\active\protected\def^^^^00a0{\leavevmode\nobreak\ }
\usepackage{cmap}
\usepackage{fontspec}
\defaultfontfeatures[\rmfamily,\sffamily,\ttfamily]{}
\usepackage{amsmath,amssymb,amstext}
\usepackage{polyglossia}
\setmainlanguage{english}



\setmainfont{FreeSerif}[
  Extension      = .otf,
  UprightFont    = *,
  ItalicFont     = *Italic,
  BoldFont       = *Bold,
  BoldItalicFont = *BoldItalic
]
\setsansfont{FreeSans}[
  Extension      = .otf,
  UprightFont    = *,
  ItalicFont     = *Oblique,
  BoldFont       = *Bold,
  BoldItalicFont = *BoldOblique,
]
\setmonofont{FreeMono}[
  Extension      = .otf,
  UprightFont    = *,
  ItalicFont     = *Oblique,
  BoldFont       = *Bold,
  BoldItalicFont = *BoldOblique,
]



\usepackage[Bjarne]{fncychap}
\usepackage[,numfigreset=1,mathnumfig]{sphinx}

\fvset{fontsize=\small}
\usepackage{geometry}


% Include hyperref last.
\usepackage{hyperref}
% Fix anchor placement for figures with captions.
\usepackage{hypcap}% it must be loaded after hyperref.
% Set up styles of URL: it should be placed after hyperref.
\urlstyle{same}


\usepackage{sphinxmessages}



        % Start of preamble defined in sphinx-jupyterbook-latex %
         \usepackage[Latin,Greek]{ucharclasses}
        \usepackage{unicode-math}
        % fixing title of the toc
        \addto\captionsenglish{\renewcommand{\contentsname}{Contents}}
        \hypersetup{
            pdfencoding=auto,
            psdextra
        }
        % End of preamble defined in sphinx-jupyterbook-latex %
        

\title{Horsefly River WCRP}
\date{Dec 20, 2022}
\release{}
\author{Canadian Wildlife Federation}
\newcommand{\sphinxlogo}{\vbox{}}
\renewcommand{\releasename}{}
\makeindex
\begin{document}

\pagestyle{empty}
\sphinxmaketitle
\pagestyle{plain}
\sphinxtableofcontents
\pagestyle{normal}
\phantomsection\label{\detokenize{intro::doc}}
\begin{sphinxuseclass}{cell}
\begin{sphinxuseclass}{tag_remove-input}
\begin{sphinxuseclass}{tag_remove-output}
\end{sphinxuseclass}
\end{sphinxuseclass}
\end{sphinxuseclass}


\sphinxAtStartPar
\sphinxincludegraphics{{spawn.jpg}.jpg}

\sphinxAtStartPar
\sphinxstylestrong{Canadian Wildlife Federation}

\sphinxAtStartPar
\sphinxstylestrong{350 Michael Cowpland Drive}

\sphinxAtStartPar
\sphinxstylestrong{Kanata, Ontario K2M 2W1}

\sphinxAtStartPar
\sphinxstylestrong{Telephone: 1\sphinxhyphen{}877\sphinxhyphen{}599\sphinxhyphen{}5777 | 613\sphinxhyphen{}599\sphinxhyphen{}9594}

\sphinxAtStartPar
\sphinxstylestrong{\sphinxhref{http://www.cwf-fcf.org}{www.cwf\sphinxhyphen{}fcf.org}}

\sphinxAtStartPar
\sphinxstylestrong{© 2022}

\sphinxAtStartPar
Suggested Citation:

\sphinxAtStartPar
Mazany\sphinxhyphen{}Wright, N., S. M. Norris, J. Noseworthy, B. Rebellato, S. Sra, and N. W. R. Lapointe. 2022. Horsefly River Watershed Connectivity Remediation Plan: 2021\sphinxhyphen{} 2040. Canadian Wildlife Federation. Ottawa, Ontario, Canada.

\sphinxAtStartPar
\sphinxhref{https://github.com/Canadian-Wildlife-Federation/Horsefly-WCRP/raw/master/Tutorials/Jupyter\_Book/mynewbook/\_build/pdf/book.pdf}{Download full PDF here!}

\begin{DUlineblock}{0em}
\item[] \sphinxstylestrong{\Large Acknowledgements}
\end{DUlineblock}

\sphinxAtStartPar
This plan represents the culmination of a collaborative planning process undertaken in the Horsefly River watershed over many months of work with a multi\sphinxhyphen{}partner planning team of individuals and groups passionate about the conservation and restoration of freshwater ecosystems and the species they support. Plan development was funded by the BC Salmon Restoration and Innovation Fund, Canada Nature Fund for Aquatic Species at Risk, and the RBC Bluewater Project. We were fortunate to benefit from the feedback, guidance, and wisdom of many groups and individuals who volunteered their time throughout this process — this publication would not have been possible without the engagement of our partners and the planning team (see \hyperref[\detokenize{Planning:table1}]{Fig.\@ \ref{\detokenize{Planning:table1}}}).

\sphinxAtStartPar
We recognize the incredible fish passage and connectivity work that has occurred in the Horsefly River watershed to date, and we are excited to continue partnering with local groups and organizations to build upon existing initiatives and provide a road map to push connectivity remediation forward over the next 20 years and beyond.

\sphinxAtStartPar
The Canadian Wildlife Federation recognizes that the lands and waters that form the basis of this plan are the traditional unceded territory of the Northern Secwepemc people. We are grateful for the opportunity to learn from the stewards of this land and work together to benefit Pacific Salmon. A special thank you to Nishitha Singi for sharing the traditional Secwepemctsín names used in this plan.

\begin{sphinxuseclass}{cell}
\begin{sphinxuseclass}{tag_remove-input}
\end{sphinxuseclass}
\end{sphinxuseclass}
\sphinxstepscope


\chapter{Plan Purpose, Apporach and Scope}
\label{\detokenize{Planning:plan-purpose-apporach-and-scope}}\label{\detokenize{Planning::doc}}
\sphinxAtStartPar
The following Watershed Connectivity Remediation Plan (WCRP) represents the culmination of a one\sphinxhyphen{}year collaborative planning effort, including field assessments, the overall aim of which is to build collaborative partnerships within the Horsefly River watershed to improve connectivity for anadromous salmon and the livelihoods that they support, including the continued sustenance, cultural, and ceremonial needs of the Northern Secwépemc people. This 20\sphinxhyphen{}year plan was developed to identify priority actions that the Horsefly River WCRP planning team (see \hyperref[\detokenize{Planning:table1}]{Fig.\@ \ref{\detokenize{Planning:table1}}} for a list of team members) will undertake between 2021\sphinxhyphen{}2040 to conserve and restore fish passage in the watershed, through crossing remediation, lateral barrier remediation, dam remediation, and barrier prevention strategies.

\sphinxAtStartPar
WCRPs are long\sphinxhyphen{}term, actionable plans that blend local stakeholder and rightsholder knowledge with innovative GIS analyses to gain a shared understanding of where remediation efforts will have the greatest benefit for anadromous salmon. The planning process is inspired by the \sphinxhref{https://cmp-openstandards.org/wp-content/uploads/2020/07/CMP-Open-Standards-for-the-Practice-of-Conservation-v4.0.pdf}{Conservation Standards} (v.4.0), which is a conservation planning framework that allows planning teams to systematically identify, implement, and monitor strategies to apply the most effective solutions to high priority conservation problems. There is a rich history of connectivity and fish habitat planning and remediation work in the Horsefly River watershed that this WCRP builds upon, including work undertaken by the BC Fish Passage Technical Working Group, the Northern Secwepemc te Qelmucw (NStQ) and member communities, the Horsefly River Roundtable, and other local organizations ({[}\hyperlink{cite.references:id17}{Ltd., 2018}{]}; S. Hocquard, Steve Hocquard Consulting, pers. comm.).

\sphinxAtStartPar
The planning team compiled existing barrier location and assessment data, habitat data, and previously identified priorities, and combined this with local and Indigenous knowledge to create a strategic watershed\sphinxhyphen{}scale plan to improve connectivity. To expand on this work the Horsefly River WCRP planning team applied the WCRP planning framework to define the “thematic” scope of freshwater connectivity and refine the “geographic” scope to identify only those portions of the watershed where barrier prioritization will be conducted, and subsequent remediation efforts will take place. Additionally, the team selected target fish species, assessed their current connectivity status in the watershed, defined concrete goals for gains in connectivity, and developed a priority list of barriers for remediation to achieve those goals. Field assessments were completed for 20 longitudinal barriers on the preliminary barrier list during the summer of 2021, followed by a series of WCRP Update Workshops in winter 2021. The aim of these workshops was for the team to receive updates on progress made during the field season, review assessment results and identify priority barriers, revise the connectivity status assessment and goals, and update the Operational Plan for 2022. While the current version of this plan is based on the best\sphinxhyphen{}available information at the time of publishing, WCRPs are intended to be “living plans” that are updated regularly as new information becomes available, or if local priorities and contexts change. As such, this document should be interpreted as a current snap\sphinxhyphen{}shot in time, and future iterations of this WCRP will build upon the material presented in this plan to continuously improve barrier remediation for migratory fish in the Horsefly River watershed. For more information on how WCRPs are developed, see {[}\hyperlink{cite.references:id14}{Mazany\sphinxhyphen{}Wright \sphinxstyleemphasis{et al.}, 2021}{]}.


\chapter{Vision Statement}
\label{\detokenize{Planning:vision-statement}}
\sphinxAtStartPar
Healthy, well\sphinxhyphen{}connected streams and rivers within the Horsefly River watershed support thriving populations of migratory fish, improving the overall ecosystem health of the watershed. In turn, these fish provide the continued sustenance, cultural, and ceremonial needs of the Northern Secwépemc people, as they have since time immemorial. Both residents and visitors to the watershed work together to mitigate the negative effects of anthropogenic aquatic barriers, improving the resiliency of streams and rivers for the benefit and appreciation of all.


\chapter{Planning Team}
\label{\detokenize{Planning:planning-team}}
\begin{sphinxuseclass}{cell}
\begin{sphinxuseclass}{tag_remove-input}
\begin{sphinxuseclass}{tag_remove-output}
\end{sphinxuseclass}
\end{sphinxuseclass}
\end{sphinxuseclass}
\begin{figure}[htbp]
\centering
\capstart
\begin{sphinxVerbatimOutput}

\begin{sphinxuseclass}{cell_output}
\begin{sphinxVerbatim}[commandchars=\\\{\}]
\PYGZlt{}pandas.io.formats.style.Styler at 0x2072801bbc8\PYGZgt{}
\end{sphinxVerbatim}

\end{sphinxuseclass}\end{sphinxVerbatimOutput}
\caption{\sphinxstyleemphasis{Horsefly River watershed WCRP planning team members. Planning team members contributed to the development of this plan by participating in a series of workshops and document and data review. The plan was generated based on the input and feedback of the local groups and organizations listed in this table.}}\label{\detokenize{Planning:table1}}\end{figure}


\chapter{Key Actors}
\label{\detokenize{Planning:key-actors}}
\begin{sphinxuseclass}{cell}
\begin{sphinxuseclass}{tag_remove-input}
\begin{sphinxuseclass}{tag_remove-output}
\end{sphinxuseclass}
\end{sphinxuseclass}
\end{sphinxuseclass}
\begin{figure}[htbp]
\centering
\capstart
\begin{sphinxVerbatimOutput}

\begin{sphinxuseclass}{cell_output}
\begin{sphinxVerbatim}[commandchars=\\\{\}]
\PYGZlt{}pandas.io.formats.style.Styler at 0x2072e2acd88\PYGZgt{}
\end{sphinxVerbatim}

\end{sphinxuseclass}\end{sphinxVerbatimOutput}
\caption{\sphinxstyleemphasis{Additional Key Actors in the Horsefly River watershed. Key Actors are the individuals, groups, and/or organizations, outside of the planning team, with influence and relevant experience in the watershed, whose engagement will be critical for the successful implementation of this WCRP.}}\label{\detokenize{Planning:table2}}\end{figure}


\chapter{Project Scope}
\label{\detokenize{Planning:project-scope}}
\sphinxAtStartPar
Connectivity is a critical component of freshwater ecosystems that encompasses a variety of factors related to ecosystem structure and function, such as the ability of aquatic organisms to disperse and/or migrate, the transportation of energy and matter (e.g., nutrient cycling and sediment flows), and temperature regulation {[}\hyperlink{cite.references:id24}{Seliger and Zeiringer, 2018}{]}. Though each of these factors are important when considering the health of a watershed, for the purposes of this WCRP the term “connectivity” is defined as the degree to which aquatic organisms can disperse and/or migrate freely through freshwater systems. Within this context, connectivity is primarily constrained by physical barriers, including anthropogenic infrastructure such as dams, weirs, and stream crossings, and natural features such as waterfalls and debris flows. This plan is intended to focus on the direct remediation and prevention of localized, physical barriers instead of the broad land\sphinxhyphen{}use patterns that are causing chronic connectivity issues in the watershed. The planning team decided that the primary focus of this WCRP is addressing barriers to both longitudinal connectivity (i.e., along the upstream\sphinxhyphen{}downstream plane) and lateral connectivity (i.e., connectivity between the mainstem and adjacent riparian wetlands and floodplains) due to the importance of maintaining fish passage to spawning, rearing, and overwintering habitat in the watershed.

\begin{figure}[htbp]
\centering
\capstart

\noindent\sphinxincludegraphics[width=1000\sphinxpxdimen,height=400\sphinxpxdimen]{{figure1}.png}
\caption{\sphinxstyleemphasis{The primary geographic scope — the Horsefly River watershed — located in the Fraser River system.}}\label{\detokenize{Planning:fig1}}\end{figure}

\sphinxAtStartPar
The primary geographic scope of this WCRP is the Horsefly River watershed, located in the upper Fraser River drainage basin in central British Columbia (\hyperref[\detokenize{Planning:fig1}]{Fig.\@ \ref{\detokenize{Planning:fig1}}}). The scope constitutes the Horsefly River “watershed group” as defined by the \sphinxhref{https://catalogue.data.gov.bc.ca/dataset/freshwater-atlas-watershed-groups}{British Columbia Freshwater Atlas} (FWA). A consistent spatial framework was necessary to undertake a watershed selection process at the provincial scale to identify target watersheds to improve connectivity for salmon. The Horsefly River watershed was identified by the BC Fish Passage Restoration Initiative as one of four target watersheds for WCRP development {[}\hyperlink{cite.references:id13}{Mazany\sphinxhyphen{}Wright \sphinxstyleemphasis{et al.}, 2021}{]}. The Horsefly River watershed has a drainage area of 276,603 ha, spanning from the Quesnel Highlands in the southeast to the confluence with Quesnel Lake in the northwest. Culturally and economically important populations of Chinook Salmon, Coho Salmon, and Sockeye Salmon are all found in the watershed, which historically supported Indigenous sustenance and trading economies (\hyperref[\detokenize{Planning:table3}]{Fig.\@ \ref{\detokenize{Planning:table3}}}; {[}\hyperlink{cite.references:id28}{Nation., 2021}{]}, {[}\hyperlink{cite.references:id31}{Nation., 2021}{]}).

\begin{sphinxuseclass}{cell}
\begin{sphinxuseclass}{tag_remove-input}
\begin{sphinxuseclass}{tag_remove-output}
\end{sphinxuseclass}
\end{sphinxuseclass}
\end{sphinxuseclass}
\begin{figure}[htbp]
\centering
\capstart
\begin{sphinxVerbatimOutput}

\begin{sphinxuseclass}{cell_output}
\begin{sphinxVerbatim}[commandchars=\\\{\}]
\PYGZlt{}pandas.io.formats.style.Styler at 0x2072e308408\PYGZgt{}
\end{sphinxVerbatim}

\end{sphinxuseclass}\end{sphinxVerbatimOutput}
\caption{\sphinxstyleemphasis{Target fish species in the Horsefly River watershed. The Secwepemctsín and Western common and scientific species names are provided.}}\label{\detokenize{Planning:table3}}\end{figure}

\sphinxAtStartPar
The Horsefly River watershed comprises parts of Secwepemcúl’ecw, the traditional territory of the Northern Secwepemc te Qelmucw (NStQ), represented by the Northern Shuswap Tribal Council and four member communities or autonomous nations:
\begin{itemize}
\item {} 
\sphinxAtStartPar
Xatśūll Cmetem’ (Soda Creek First Nations)

\item {} 
\sphinxAtStartPar
Stswēceḿc Xgāt’tem (Canoe Creek/Dog Creek First Nations)

\item {} 
\sphinxAtStartPar
T’ēxelc (Williams Lake First Nation)

\item {} 
\sphinxAtStartPar
Tsq’ēsceń (Canim Lake First Nation)

\end{itemize}

\sphinxAtStartPar
The geographic scope of this WCRP was further refined by identifying “potentially accessible” stream segments, which are defined as streams that target species should be able to access in the absence of anthropogenic barriers (\hyperref[\detokenize{Planning:fig2}]{Fig.\@ \ref{\detokenize{Planning:fig2}}}). Potentially accessible stream segments were spatially delineated using fish species observation and distribution data, as well as data on “exclusionary points”. These include waterfalls greater than 5 m in height, gradient barriers based on species\sphinxhyphen{}specific swimming abilities, and watershed exclusion areas, which are portions of the watershed where barrier remediation efforts should not occur. These maps were explored by the planning team to incorporate additional local knowledge, ensure accuracy, and finalize the constraints on potentially accessible stream segments. The planning team identified certain tributaries to the mainstem Horsefly River as “watershed exclusion areas”, which were excluded from further consideration under this plan, due to intermittent or insufficient flows to support restoring connectivity for the target species. The geographic scope was further refined based on several confirmed impassable waterfalls and modelled gradient barriers. Specifically, there are two impassable waterfalls that severely limit potentially accessible habitat: one on the mainstem Horsefly River approximately 4 km upstream of the confluence with McKinley Creek, and the second on Moffat Creek approximately 5 km upstream from where it flows into the Horsefly River. All stream segments not identified as potentially accessible were removed from the scope for further consideration. The “constrained geographic scope” formed the foundation for all subsequent analyses and planning steps, including mapping and modelling useable habitat types, quantifying the current connectivity status, goal setting, and action planning {[}\hyperlink{cite.references:id12}{Mazany\sphinxhyphen{}Wright \sphinxstyleemphasis{et al.}, 2021}{]}.

\begin{figure}[htbp]
\centering
\capstart

\noindent\sphinxincludegraphics[width=1000\sphinxpxdimen,height=400\sphinxpxdimen]{{figure2}.png}
\caption{\sphinxstyleemphasis{Potentially accessible stream segments within the Horsefly River watershed. These do not represent useable habitat types, but rather identifies the stream segments within which habitat modelling and barrier mapping and prioritization was undertaken.}}\label{\detokenize{Planning:fig2}}\end{figure}


\chapter{Target species}
\label{\detokenize{Planning:target-species}}
\sphinxAtStartPar
Target species represent the ecologically and culturally important species for which habitat connectivity is being conserved and/or restored in the watershed. In the Horsefly River watershed, the planning team selected Anadromous Salmon as the target species group, which comprises Chinook Salmon, Coho Salmon, and Sockeye Salmon. The selection of these target species was driven primarily by the targets species of the primary fund supporting this planning work.


\section{Anadromous Salmonids}
\label{\detokenize{Planning:anadromous-salmonids}}
\sphinxAtStartPar
Anadromous salmon are cultural and ecological keystone species that contribute to productive ecosystems by contributing marine\sphinxhyphen{}derived nutrients to the watershed and forming an important food source for other species. Salmon species are sacred to the NStQ, having sustained life, trading economies, and culture since time immemorial ({[}\hyperlink{cite.references:id28}{Nation., 2021}{]}, {[}\hyperlink{cite.references:id31}{Nation., 2021}{]}, N. Singi pers. comm.). The stewardship of the resources and fisheries in their traditional territories are imbued in the spirit of the NStQ through a symbiotic relationship based on respect – the NStQ never take more salmon than is needed and there is no waste. The entirety of the salmon is used \sphinxhyphen{} smoked and dried to sustain the NStQ through the winter months, the roe harvested for consumption, salmon oil rendered to be stored and traded, and the skin used to store the oil ({[}\hyperlink{cite.references:id29}{Wilson \sphinxstyleemphasis{et al.}, 1998}{]}, {[}\hyperlink{cite.references:id31}{Nation., 2021}{]}, N. Singi pers. comm.). The salmon runs begin to return to the Horsefly River watershed in early August, and the NStQ traditionally celebrate and feast at this time. The harvest of the salmon strengthens the cultural connection to the land and the waters, providing an important food source for communities and the opportunity to pass knowledge and ceremony to future generations through fishing and fish processing ({[}\hyperlink{cite.references:id28}{Nation., 2021}{]}, {[}\hyperlink{cite.references:id31}{Nation., 2021}{]}).

\sphinxAtStartPar
Anadromous salmon populations in the Horsefly River watershed have declined significantly in the past few decades, with the populations of all three focal species being listed as Threatened or Endangered by the Committee On the Status of Endangered Wildlife In Canada (COSEWIC). This has been exacerbated by the Big Bar landslide on the Fraser River in 2019, leading the four NStQ communities to voluntarily close the salmon fishery from 2019\sphinxhyphen{}2022. The stewardship of their waters continues through the work of the NStQ member communities and the Northern Shuswap Tribal Council. See Appendix A for maps of modelled anadromous salmon habitat in the Horsefly River Watershed.


\section{Chinook Salmon | Kekèsu | Oncorhynchus tshawytscha}
\label{\detokenize{Planning:chinook-salmon-kekesu-oncorhynchus-tshawytscha}}
\begin{sphinxuseclass}{cell}
\begin{sphinxuseclass}{tag_remove-input}\begin{sphinxVerbatimOutput}

\begin{sphinxuseclass}{cell_output}
\begin{sphinxVerbatim}[commandchars=\\\{\}]
\PYGZlt{}pandas.io.formats.style.Styler at 0x2072e345ec8\PYGZgt{}
\end{sphinxVerbatim}

\end{sphinxuseclass}\end{sphinxVerbatimOutput}

\end{sphinxuseclass}
\end{sphinxuseclass}
\begin{sphinxuseclass}{cell}
\begin{sphinxuseclass}{tag_remove-input}
\begin{sphinxuseclass}{tag_remove-output}
\end{sphinxuseclass}
\end{sphinxuseclass}
\end{sphinxuseclass}
\begin{figure}[htbp]
\centering
\capstart
\begin{sphinxVerbatimOutput}

\begin{sphinxuseclass}{cell_output}
\begin{sphinxVerbatim}[commandchars=\\\{\}]
\PYGZlt{}pandas.io.formats.style.Styler at 0x2072e3487c8\PYGZgt{}
\end{sphinxVerbatim}

\end{sphinxuseclass}\end{sphinxVerbatimOutput}
\caption{\sphinxstyleemphasis{Chinook Salmon population assessments in the Horsefly River watershed. Conservation Unit assessments were undertaken by the \sphinxhref{https://www.salmonexplorer.ca/\#\%21/fraser/chinook/middle-fraser-river-spring-5-2}{Pacific Salmon Foundation} (\sphinxhref{https://salmonwatersheds.ca/libraryfiles/lib\_459.pdf}{2020}). Designated Unit assessments were undertaken by \sphinxhref{https://www.canada.ca/en/environment-climate-change/services/species-risk-public-registry/cosewic-assessments-status-reports/chinook-salmon-2018.html}{COSEWIC} (2018).}}\label{\detokenize{Planning:table4}}\end{figure}

\sphinxAtStartPar
Chinook Salmon are the first to return each year, usually in early August {[}\hyperlink{cite.references:id11}{DFO, 1991}{]}, and have the most limited distribution within the watershed. Known spawning occurs in parts of the Horsefly River mainstem above the confluence with the Little Horsefly River and throughout McKinley Creek as far as Elbow Lake ({[}\hyperlink{cite.references:id11}{DFO, 1991}{]}, S. Hocquard, pers. comm.). Important rearing systems include Patenaude Creek, Kroener Creek, Black Creek, Woodjam Creek, Deerhorn Creek, and Wilmot Creek (S. Hocquard, pers. comm.).


\section{Coho Salmon | Sxeyqs | Oncorhynchus kisutch}
\label{\detokenize{Planning:coho-salmon-sxeyqs-oncorhynchus-kisutch}}
\begin{sphinxuseclass}{cell}
\begin{sphinxuseclass}{tag_remove-input}\begin{sphinxVerbatimOutput}

\begin{sphinxuseclass}{cell_output}
\begin{sphinxVerbatim}[commandchars=\\\{\}]
\PYGZlt{}pandas.io.formats.style.Styler at 0x2072e184408\PYGZgt{}
\end{sphinxVerbatim}

\end{sphinxuseclass}\end{sphinxVerbatimOutput}

\end{sphinxuseclass}
\end{sphinxuseclass}
\begin{sphinxuseclass}{cell}
\begin{sphinxuseclass}{tag_remove-input}
\begin{sphinxuseclass}{tag_remove-output}
\end{sphinxuseclass}
\end{sphinxuseclass}
\end{sphinxuseclass}
\begin{figure}[htbp]
\centering
\capstart
\begin{sphinxVerbatimOutput}

\begin{sphinxuseclass}{cell_output}
\begin{sphinxVerbatim}[commandchars=\\\{\}]
\PYGZlt{}pandas.io.formats.style.Styler at 0x2072d5cbfc8\PYGZgt{}
\end{sphinxVerbatim}

\end{sphinxuseclass}\end{sphinxVerbatimOutput}
\caption{\sphinxstyleemphasis{Coho Salmon population assessments in the Horsefly River watershed. Conservation Unit assessments were undertaken by the \sphinxhref{https://www.salmonexplorer.ca/\#\%21/fraser/chinook/middle-fraser-river-spring-5-2}{Pacific Salmon Foundation} (\sphinxhref{https://salmonwatersheds.ca/libraryfiles/lib\_459.pdf}{2020}). Designated Unit assessments were undertaken by \sphinxhref{https://www.canada.ca/en/environment-climate-change/services/species-risk-public-registry/cosewic-assessments-status-reports/chinook-salmon-2018.html}{COSEWIC} (2016).}}\label{\detokenize{Planning:table5}}\end{figure}

\sphinxAtStartPar
Coho Salmon are the most widely distributed of the three focal species in the watershed, with the ability to migrate into smaller, upper tributary systems {[}\hyperlink{cite.references:id11}{DFO, 1991}{]}. Spawning occurs in the Little Horsefly River between Gruhs Lake and Horsefly Lake, McKinley Creek below McKinley Lake, Woodjam Creek, Patenaude Creek, Tisdall Creek, and Black Creek. Rearing fry and juveniles have been observed in the Little Horsefly River, Patenaude Creek, and McKinley Creek up to Bosk Lake ({[}\hyperlink{cite.references:id11}{DFO, 1991}{]}, S. Hocquard pers. comm.).


\section{Sockeye Salmon | Sqlelten7ùwi | Oncorhynchus nerka}
\label{\detokenize{Planning:sockeye-salmon-sqlelten7uwi-oncorhynchus-nerka}}
\begin{sphinxuseclass}{cell}
\begin{sphinxuseclass}{tag_remove-input}\begin{sphinxVerbatimOutput}

\begin{sphinxuseclass}{cell_output}
\begin{sphinxVerbatim}[commandchars=\\\{\}]
\PYGZlt{}pandas.io.formats.style.Styler at 0x2072e350508\PYGZgt{}
\end{sphinxVerbatim}

\end{sphinxuseclass}\end{sphinxVerbatimOutput}

\end{sphinxuseclass}
\end{sphinxuseclass}
\begin{sphinxuseclass}{cell}
\begin{sphinxuseclass}{tag_remove-input}
\begin{sphinxuseclass}{tag_remove-output}
\end{sphinxuseclass}
\end{sphinxuseclass}
\end{sphinxuseclass}
\begin{figure}[htbp]
\centering
\capstart
\begin{sphinxVerbatimOutput}

\begin{sphinxuseclass}{cell_output}
\begin{sphinxVerbatim}[commandchars=\\\{\}]
\PYGZlt{}pandas.io.formats.style.Styler at 0x2072e37a2c8\PYGZgt{}
\end{sphinxVerbatim}

\end{sphinxuseclass}\end{sphinxVerbatimOutput}
\caption{\sphinxstyleemphasis{Sockeye Salmon population assessments in the Horsefly River watershed. Conservation Unit assessments were undertaken by the \sphinxhref{https://www.salmonexplorer.ca/\#\%21/fraser/chinook/middle-fraser-river-spring-5-2}{Pacific Salmon Foundation} (\sphinxhref{https://salmonwatersheds.ca/libraryfiles/lib\_459.pdf}{2020}). Designated Unit assessments were undertaken by \sphinxhref{https://www.canada.ca/en/environment-climate-change/services/species-risk-public-registry/cosewic-assessments-status-reports/chinook-salmon-2018.html}{COSEWIC} (2017).}}\label{\detokenize{Planning:table6}}\end{figure}

\sphinxAtStartPar
Sockeye Salmon have historically been the most abundant of the three focal species in the watershed, though the population has seen significant declines in recent years ({[}\hyperlink{cite.references:id11}{DFO, 1991}{]}, S. Hocquard pers. comm.). Sockeye Salmon spawning is known to occur throughout the Horsefly River (up to the impassable falls), in the Little Horsefly River between Gruhs Lake and Horsefly Lake, Moffat Creek (up to the impassible falls), and McKinley Creek up to Elbow Lake ({[}\hyperlink{cite.references:id19}{Pacific\sphinxhyphen{}Salmon\sphinxhyphen{}Foundation, 2020}{]}, {[}\hyperlink{cite.references:id11}{DFO, 1991}{]}, S. Hocquard pers. comm.). Additionally, a spawning channel aimed at enhancing the Sockeye Salmon population was constructed by Fisheries and Oceans Canada in 1989 {[}\hyperlink{cite.references:id11}{DFO, 1991}{]}. Currently, there are no Sockeye Salmon rearing in the Horsefly River watershed – all emergent fry migrate down to Quesnel Lake.

\sphinxstepscope


\chapter{Connectivity Status Assessment and Action Plan}
\label{\detokenize{EcoAttributes:connectivity-status-assessment-and-action-plan}}\label{\detokenize{EcoAttributes::doc}}
\begin{sphinxuseclass}{cell}
\begin{sphinxuseclass}{tag_remove-input}
\begin{sphinxuseclass}{tag_remove-output}
\end{sphinxuseclass}
\end{sphinxuseclass}
\end{sphinxuseclass}
\sphinxAtStartPar
The planning team devised two Key Ecological Attributes (KEAs) and associated indicators to assess the current connectivity status of the watershed – Accessible Habitat and Accessible Overwintering Habitat (\hyperref[\detokenize{EcoAttributes:table7}]{Fig.\@ \ref{\detokenize{EcoAttributes:table7}}}). KEAs are the key aspects of anadromous salmon ecology that are being targeted by this WCRP. The connectivity status of Anadromous Salmon was used to establish goals to improve habitat connectivity in the watershed and will be the baseline against which progress is tracked over time.

\sphinxAtStartPar
The current connectivity status assessment relies on GIS analyses to map known and modelled barriers to fish passage, identify stream reaches that have potential spawning and rearing habitat, estimate the proportion of habitat that is currently accessible to target species, and prioritize barriers for field assessment that would provide the greatest gains in connectivity. To support a flexible prioritization framework to identify priority barriers in the watershed, two assumptions are made: 1) any modelled (i.e., passability status is unknown) or partial barriers are treated as complete barriers to passage and 2) the habitat modelling is binary, it does not assign any habitat quality values. As such, the current connectivity status will be refined over time as more data on habitat and barriers are collected. For more detail on how the connectivity status assessments were conducted, see Appendix B.

\begin{sphinxuseclass}{cell}
\begin{sphinxuseclass}{tag_remove-input}\begin{sphinxVerbatimOutput}

\begin{sphinxuseclass}{cell_output}
\begin{sphinxVerbatim}[commandchars=\\\{\}]
\PYGZlt{}pandas.io.formats.style.Styler at 0x29c3ae3ba48\PYGZgt{}
\end{sphinxVerbatim}

\end{sphinxuseclass}\end{sphinxVerbatimOutput}

\end{sphinxuseclass}
\end{sphinxuseclass}
\sphinxAtStartPar
\sphinxstylestrong{Comments}: Indicator rating definitions are based on the consensus decisions of the planning team, including the decision not to define Fair. The current status is based on the CWF Barrier Prioritization Model output, which is current as of March 2022.

\begin{sphinxuseclass}{cell}
\begin{sphinxuseclass}{tag_remove-input}
\begin{sphinxuseclass}{tag_remove-output}
\end{sphinxuseclass}
\end{sphinxuseclass}
\end{sphinxuseclass}
\sphinxAtStartPar
\sphinxstylestrong{Comments:} No baseline data exists on the extent of overwintering habitat in the watershed. A priority action is included in the Operational Plan (strategy 2.3) to develop a habitat layer, and this will be used to inform this connectivity status assessment in the future.

\begin{figure}[htbp]
\centering
\capstart
\begin{sphinxVerbatimOutput}

\begin{sphinxuseclass}{cell_output}
\begin{sphinxVerbatim}[commandchars=\\\{\}]
\PYGZlt{}pandas.io.formats.style.Styler at 0x29c3b7c4b08\PYGZgt{}
\end{sphinxVerbatim}

\end{sphinxuseclass}\end{sphinxVerbatimOutput}
\caption{\sphinxstyleemphasis{Connectivity status assessment for (a) linear habitat (spawning and rearing) and (b) overwintering habitat in the Horsefly River watershed. The Available Habitat KEA is evaluated by dividing the length of linear habitat that is currently accessible to target species by the total length of all linear habitat in the watershed. The Available Overwintering Habitat KEA is evaluated as the sum of all areal overwintering habitat that is accessible to target species.}}\label{\detokenize{EcoAttributes:table7}}\end{figure}


\chapter{Barrier Types}
\label{\detokenize{EcoAttributes:barrier-types}}
\sphinxAtStartPar
The following table highlights which barrier types pose the greatest threat to anadromous salmon in the watershed. The results of this assessment were used to inform the subsequent planning steps, as well as to identify knowledge gaps where there is little spatial data to inform the assessment for a specific barrier type.

\begin{sphinxuseclass}{cell}
\begin{sphinxuseclass}{tag_remove-input}
\begin{sphinxuseclass}{tag_remove-output}
\end{sphinxuseclass}
\end{sphinxuseclass}
\end{sphinxuseclass}
\begin{figure}[htbp]
\centering
\capstart
\begin{sphinxVerbatimOutput}

\begin{sphinxuseclass}{cell_output}
\begin{sphinxVerbatim}[commandchars=\\\{\}]
\PYGZlt{}pandas.io.formats.style.Styler at 0x29c3e2d4388\PYGZgt{}
\end{sphinxVerbatim}

\end{sphinxuseclass}\end{sphinxVerbatimOutput}
\caption{\sphinxstyleemphasis{Barrier Types in the Horsefly River watershed and barrier rating assessment results. For each barrier type listed, “Extent refers to the proportion of anadromous salmon habitat that is being blocked by that barrier type, “Severity” is the proportion of structures for each barrier type that are known to block passage for target species based on field assessments, and “Irreversibility” is the degree to which the effects of a barrier type can be reversed and connectivity restored. The amount of habitat blocked used in this exercise is a representation of total amount of combined spawning and rearing habitat. All ratings in this table have been updated from version 1.0 to version 2.0 of the Horsefly River Watershed Connectivity Remediation Plan based on the most recent field assessments.}}\label{\detokenize{EcoAttributes:table8}}\end{figure}

\begin{sphinxuseclass}{cell}
\begin{sphinxuseclass}{tag_remove-input}
\begin{sphinxuseclass}{tag_remove-output}
\end{sphinxuseclass}
\end{sphinxuseclass}
\end{sphinxuseclass}

\section{Small Dams (<3 m height)}
\label{\detokenize{EcoAttributes:small-dams-3-m-height}}
\sphinxAtStartPar
There are \DUrole{pasted-text}{9} mapped small dams on “potentially accessible” stream segments in the watershed, blocking a total of \DUrole{pasted-text}{8.09} km (\textasciitilde{}\DUrole{pasted-text}{23}\% of the total blocked habitat) of modelled spawning and rearing habitat for anadromous salmon, resulting in a medium extent. The extent rating of these structures was confirmed by the planning team. There are two known fish\sphinxhyphen{}passage structures in the watershed, including on the dam at the outlet of McKinley Lake. The remaining dams likely block passage for anadromous salmon and would require significant resources to remediate. However, due to the limited extent of dams in the watershed, a final pressure rating of Medium was assigned. Four small dams were identified on the priority barrier list (see Appendix B). Three of the dams require further assessment and confirmation of upstream habitat quality, and the dam observed at the outlet of Kwun Lake does not exist.

\begin{sphinxuseclass}{cell}
\begin{sphinxuseclass}{tag_remove-input}
\begin{sphinxuseclass}{tag_remove-output}
\end{sphinxuseclass}
\end{sphinxuseclass}
\end{sphinxuseclass}

\section{Road\sphinxhyphen{}stream Crossings}
\label{\detokenize{EcoAttributes:road-stream-crossings}}
\sphinxAtStartPar
Road\sphinxhyphen{}stream crossings are the most abundant barrier type in the watershed, with \DUrole{pasted-text}{244} assessed and modelled crossings located on stream segments with modelled habitat. Demographic road crossings (highways, municipal, and paved roads) block \DUrole{pasted-text}{7.31} km of habitat (\textasciitilde{}\DUrole{pasted-text}{21}\% of the total blocked habitat), with \DUrole{pasted-text}{73}\% of assessed crossings having been identified as barriers to fish passage. Resource roads block \DUrole{pasted-text}{19.57} km of habitat (\textasciitilde{}\DUrole{pasted-text}{56}\%), with \DUrole{pasted-text}{60}\% of assessed crossings having been identified as barriers. The planning team felt that the data was underestimating the severity of road\sphinxhyphen{}stream crossing barriers in the watershed, and therefore decided to update the rating from High to Very High. The planning team also felt that an irreversibility rating of Medium was appropriate due to the technical complexity and resources required to remediate road\sphinxhyphen{}stream crossings.


\section{Trail\sphinxhyphen{}stream crossings}
\label{\detokenize{EcoAttributes:trail-stream-crossings}}
\sphinxAtStartPar
There is very little spatial data available on trail\sphinxhyphen{}stream crossings in the watershed, so the planning team was unable to quantify the true Extent and Severity of this barrier type. However, the planning team felt that trail\sphinxhyphen{}stream crossings are not prevalent within the watershed and that, where they do exist, they do not significantly impact passage for anadromous salmon. As most crossings will be fords or similar structures, remediation may not be required, or remediation costs associated with these barriers would be quite low. Overall, the planning team felt that the pressure rating for trail\sphinxhyphen{}stream crossings was likely Low; however, the lack of ground\sphinxhyphen{}truthed evidence to support this rating was identified as a knowledge gap within this plan.


\section{Lateral Barriers}
\label{\detokenize{EcoAttributes:lateral-barriers}}
\sphinxAtStartPar
There are numerous types of lateral barriers that potentially occur in the watershed, including dykes, berms, and linear development (i.e., road and rail lines), all of which can restrict the ability of anadromous salmon to move into floodplains, riparian wetlands, and other off\sphinxhyphen{}channel habitats. No comprehensive lateral barrier data exists within the watershed, so pressure ratings were based on qualitative local knowledge. Lateral barriers are not thought to be as prevalent as road\sphinxhyphen{} or rail\sphinxhyphen{}stream crossings but are likely very severe where they do exist. Significant lateral barriers are known to occur along the mainstem of the Horsefly River, which disconnect the mainstem river from historic floodplain and off\sphinxhyphen{}channel habitat. Overall, the planning team decided that a High pressure rating adequately captured the effect that lateral barriers are having on connectivity in the watershed. Work to begin quantifying and mapping lateral habitat will begin in 2022\sphinxhyphen{}23, as described in the Operational Plan under Strategy 2: Lateral barrier remediation.


\section{Natural Barriers}
\label{\detokenize{EcoAttributes:natural-barriers}}
\sphinxAtStartPar
Natural barriers to fish passage can include debris flows, log jams, sediment deposits, etc., but natural features that have always restricted fish passage (e.g., waterfalls) are not considered under this barrier type. Natural barriers are difficult to include in a spatial prioritization framework due to their transient nature. The planning team identified known natural barriers that occur throughout the watershed, such as beaver dams and log jams. Generally, these natural barriers are only severe impediments to fish passage during low\sphinxhyphen{}flow years, but reduced baseflows have become more common in recent years. Based on this, the planning team felt that natural barriers will be severe most years where they exist, but are mostly reversible, resulting in an overall pressure rating of Low.


\chapter{Situation Analysis}
\label{\detokenize{EcoAttributes:situation-analysis}}
\sphinxAtStartPar
The following situation model was developed by the WCRP planning team to “map” the project context and brainstorm potential actions for implementation. Green text is used to identify actions that were selected for implementation (see Strategies \& Actions), and red text is used to identify actions that the project team has decided to exclude from the current iteration of the plan, as they were either outside of the project scope, or were deemed to be ineffective by the planning team.

\begin{figure}[htbp]
\centering
\capstart

\noindent\sphinxincludegraphics[width=1000\sphinxpxdimen,height=400\sphinxpxdimen]{{figure3}.png}
\caption{\sphinxstyleemphasis{Situation analysis developed by the planning team to identify factors that contribute to fragmentation (orange boxes), biophysical results (brown boxes), and potential strategies/actions to improve connectivity (yellow hexagons) for target species in the Horsefly River watershed.}}\label{\detokenize{EcoAttributes:directive-fig}}\end{figure}


\chapter{Goals}
\label{\detokenize{EcoAttributes:goals}}
\begin{sphinxuseclass}{cell}
\begin{sphinxuseclass}{tag_remove-input}
\begin{sphinxuseclass}{tag_remove-output}
\end{sphinxuseclass}
\end{sphinxuseclass}
\end{sphinxuseclass}
\begin{figure}[htbp]
\centering
\capstart
\begin{sphinxVerbatimOutput}

\begin{sphinxuseclass}{cell_output}
\begin{sphinxVerbatim}[commandchars=\\\{\}]
\PYGZlt{}pandas.io.formats.style.Styler at 0x29c3b74f908\PYGZgt{}
\end{sphinxVerbatim}

\end{sphinxuseclass}\end{sphinxVerbatimOutput}
\caption{\sphinxstyleemphasis{Goals to improve (1) spawning and rearing and (2) overwintering habitat connectivity for target species in the Horsefly River watershed over the lifespan of the WCRP (2021\sphinxhyphen{}2040). The goals were established through discussions with the planning team and represent the resulting desired state of connectivity in the watershed. The goals are subject to change as more information and data are collected over the course of the plan timeline (e.g., the current connectivity status is updated based on barrier field assessments).}}\label{\detokenize{EcoAttributes:table9}}\end{figure}


\chapter{Strategies \& Actions}
\label{\detokenize{EcoAttributes:strategies-actions}}
\sphinxAtStartPar
Effectiveness evaluation of identified conservation strategies and associated actions to improve connectivity for target species in the Horsefly River watershed. The planning team identified five broad strategies to implement through this WCRP, 1) crossing remediation, 2) lateral barrier remediation, 3) dam remediation, 4) barrier prevention, and 5) communication and education. Individual actions were qualitatively evaluated based on the anticipated effect each action will have on realizing on\sphinxhyphen{}the\sphinxhyphen{}ground gains in connectivity. Effectiveness ratings are based on a combination of “Feasibility and “Impact”, Feasibility is defined as the degree to which the project team can implement the action within realistic constraints (financial, time, ethical, etc.) and Impact is the degree to which the action is likely to contribute to achieving one or more of the goals established in this plan.


\section{Strategy 1: Crossing Remediation}
\label{\detokenize{EcoAttributes:strategy-1-crossing-remediation}}
\begin{sphinxuseclass}{cell}
\begin{sphinxuseclass}{tag_remove-input}\begin{sphinxVerbatimOutput}

\begin{sphinxuseclass}{cell_output}
\begin{sphinxVerbatim}[commandchars=\\\{\}]
\PYGZlt{}pandas.io.formats.style.Styler at 0x29c3b774c88\PYGZgt{}
\end{sphinxVerbatim}

\end{sphinxuseclass}\end{sphinxVerbatimOutput}

\end{sphinxuseclass}
\end{sphinxuseclass}

\section{Strategy 2: Lateral Barrier Remediation}
\label{\detokenize{EcoAttributes:strategy-2-lateral-barrier-remediation}}
\begin{sphinxuseclass}{cell}
\begin{sphinxuseclass}{tag_remove-input}\begin{sphinxVerbatimOutput}

\begin{sphinxuseclass}{cell_output}
\begin{sphinxVerbatim}[commandchars=\\\{\}]
\PYGZlt{}pandas.io.formats.style.Styler at 0x29c3e275348\PYGZgt{}
\end{sphinxVerbatim}

\end{sphinxuseclass}\end{sphinxVerbatimOutput}

\end{sphinxuseclass}
\end{sphinxuseclass}

\section{Strategy 3: Dam Remediation}
\label{\detokenize{EcoAttributes:strategy-3-dam-remediation}}
\begin{sphinxuseclass}{cell}
\begin{sphinxuseclass}{tag_remove-input}\begin{sphinxVerbatimOutput}

\begin{sphinxuseclass}{cell_output}
\begin{sphinxVerbatim}[commandchars=\\\{\}]
\PYGZlt{}pandas.io.formats.style.Styler at 0x29c3e28a408\PYGZgt{}
\end{sphinxVerbatim}

\end{sphinxuseclass}\end{sphinxVerbatimOutput}

\end{sphinxuseclass}
\end{sphinxuseclass}

\section{Strategy 4: Barrier Prevention}
\label{\detokenize{EcoAttributes:strategy-4-barrier-prevention}}
\begin{sphinxuseclass}{cell}
\begin{sphinxuseclass}{tag_remove-input}\begin{sphinxVerbatimOutput}

\begin{sphinxuseclass}{cell_output}
\begin{sphinxVerbatim}[commandchars=\\\{\}]
\PYGZlt{}pandas.io.formats.style.Styler at 0x29c3a7f7f08\PYGZgt{}
\end{sphinxVerbatim}

\end{sphinxuseclass}\end{sphinxVerbatimOutput}

\end{sphinxuseclass}
\end{sphinxuseclass}

\section{Strategy 5: Communication and Education}
\label{\detokenize{EcoAttributes:strategy-5-communication-and-education}}
\begin{sphinxuseclass}{cell}
\begin{sphinxuseclass}{tag_remove-input}\begin{sphinxVerbatimOutput}

\begin{sphinxuseclass}{cell_output}
\begin{sphinxVerbatim}[commandchars=\\\{\}]
\PYGZlt{}pandas.io.formats.style.Styler at 0x29c3e2e3448\PYGZgt{}
\end{sphinxVerbatim}

\end{sphinxuseclass}\end{sphinxVerbatimOutput}

\end{sphinxuseclass}
\end{sphinxuseclass}

\chapter{Theories of Change \& Objectives}
\label{\detokenize{EcoAttributes:theories-of-change-objectives}}
\sphinxAtStartPar
Theories of Change are explicit assumptions around how the identified actions will achieve gains in connectivity and contribute towards reaching the goals of the plan. To develop Theories of Change, the planning team developed explicit assumptions for each strategy which helped to clarify the rationale used for undertaking actions and provided an opportunity for feedback on invalid assumptions or missing opportunities. The Theories of Change are results oriented and clearly define the expected outcome. The following theory of change models were developed by the WCRP planning team to “map” the causal (“if\sphinxhyphen{}then”) progression of assumptions of how the actions within a strategy work together to achieve project goals.

\begin{figure}[htbp]
\centering
\capstart

\noindent\sphinxincludegraphics[width=1000\sphinxpxdimen,height=400\sphinxpxdimen]{{figure4}.png}
\caption{\sphinxstyleemphasis{Theory of change developed by the planning team for the actions identified under Strategy 1: Crossing Remediation in the Horsefly River watershed.}}\label{\detokenize{EcoAttributes:fig4}}\end{figure}

\begin{figure}[htbp]
\centering
\capstart

\noindent\sphinxincludegraphics[width=1000\sphinxpxdimen,height=400\sphinxpxdimen]{{figure5}.png}
\caption{\sphinxstyleemphasis{Theory of change developed by the planning team for the actions identified under Strategy 2: Lateral Barrier Remediation in the Horsefly River watershed.}}\label{\detokenize{EcoAttributes:fig5}}\end{figure}

\begin{figure}[htbp]
\centering
\capstart

\noindent\sphinxincludegraphics[width=1000\sphinxpxdimen,height=400\sphinxpxdimen]{{figure6}.png}
\caption{\sphinxstyleemphasis{Theory of change developed by the planning team for the actions identified under Strategy 3: Dam Remediation in the Horsefly River watershed.}}\label{\detokenize{EcoAttributes:fig6}}\end{figure}

\begin{figure}[htbp]
\centering
\capstart

\noindent\sphinxincludegraphics[width=1000\sphinxpxdimen,height=400\sphinxpxdimen]{{figure7}.png}
\caption{\sphinxstyleemphasis{Theory of change developed by the planning team for the actions identified under Strategy 4: Barrier Prevention in the Horsefly River watershed.}}\label{\detokenize{EcoAttributes:fig7}}\end{figure}


\chapter{Operational Plan}
\label{\detokenize{EcoAttributes:operational-plan}}
\sphinxAtStartPar
The operational plan represents a preliminary exercise undertaken by the planning team to identify the potential leads, potential participants, and estimated cost for the implementation of each action in the Horsefly River watershed. The table below summarizes individuals, groups, or organizations that the planning team felt could lead or participate in the implementation of the plan and should be interpreted as the first step in on\sphinxhyphen{}going planning and engagement to develop more detailed and sophisticated action plans for each entry in the table. The individuals, groups, and organizations listed under the “Lead(s)” or “Potential Participants” columns are those that provisionally expressed interest in participating in one of those roles or were suggested by the planning team for further engagement (denoted in bold), for those that are not members of the planning team. The leads, participants, and estimated costs in the operational plan are not binding nor an official commitment of resources, but rather provide a roadmap for future coordination and engagement to work towards implementation of the WCRP.

\begin{sphinxuseclass}{cell}
\begin{sphinxuseclass}{tag_remove-input}
\begin{sphinxuseclass}{tag_remove-output}
\end{sphinxuseclass}
\end{sphinxuseclass}
\end{sphinxuseclass}

\chapter{Funding Sources}
\label{\detokenize{EcoAttributes:funding-sources}}
\begin{sphinxuseclass}{cell}
\begin{sphinxuseclass}{tag_remove-input}
\begin{sphinxuseclass}{tag_remove-output}
\end{sphinxuseclass}
\end{sphinxuseclass}
\end{sphinxuseclass}
\sphinxstepscope


\chapter{References}
\label{\detokenize{references:references}}\label{\detokenize{references::doc}}
\sphinxstepscope


\chapter{Appendix A}
\label{\detokenize{AppendixA:appendix-a}}\label{\detokenize{AppendixA::doc}}

\section{Modelled Anadromous Salmon Habitat Maps}
\label{\detokenize{AppendixA:modelled-anadromous-salmon-habitat-maps}}
\sphinxAtStartPar
High\sphinxhyphen{}resolution PDF maps of the Horsefly River watershed and model results can be accessed \sphinxhref{https://github.com/smnorris/bcfishpass/tree/main/wcrp/pdfs}{here}. The watershed is divided into multiple maps sheets to allow for detailed examination of modelled spawning and rearing habitat, multiple barrier types, and priority barriers identified through this planning process. The locations of WCRP priority barriers and associated map sheet numbers are shown below. In each individual map sheet, priority barriers are symbolized using the following notation:

\begin{figure}[htbp]
\centering
\capstart

\noindent\sphinxincludegraphics[width=1000\sphinxpxdimen,height=400\sphinxpxdimen]{{figure8}.png}
\caption{\sphinxstyleemphasis{Horsefly River watershed overview map identifying the portions of the watershed covered by each map sheet (grey squares) and the prioritized barriers on the intermediate barrier list (orange points; see Appendix B).}}\label{\detokenize{AppendixA:fig8}}\end{figure}


\section{Connectivity Status Assessment Methods}
\label{\detokenize{AppendixA:connectivity-status-assessment-methods}}
\sphinxAtStartPar
The connectivity status assessment for anadromous salmonids in the Horsefly River watershed builds on existing connectivity modelling work undertaken by the BC Fish Passage Technical Working Group, resulting in a flexible, customizable open\sphinxhyphen{}source spatial model called “bcfishpass”. The model spatially locates known and modelled barriers to fish passage, identifies potential spawning and rearing habitat for target species, and estimates the amount of habitat that is currently accessible to target species. The model uses an adapted version of the Intrinsic Potential (IP) fish habitat modelling framework (see Sheer et al. 2009 for an overview of the IP framework). The habitat model uses two geomorphic characteristics of the stream network — channel gradient and mean annual discharge — to identify potential spawning habitat and rearing habitat for each target species. The habitat model does not attempt to definitively map each habitat type nor estimate habitat quality, but rather identifies stream segments that have high potential to support spawning or rearing habitat for each species based on the geomorphic characteristics of the segment. For more details on the connectivity and habitat model structure and parameters, please see {[}\hyperlink{cite.references:id12}{Mazany\sphinxhyphen{}Wright \sphinxstyleemphasis{et al.}, 2021}{]}. The variables and thresholds used to model potential spawning and rearing habitat for each target species are summarized in Table 15. The quantity of modelled habitat for each species was aggregated for each habitat type and represents a linear measure of potential habitat. To recognize the rearing value provided by features represented by polygons for certain species (e.g., wetlands for Coho Salmon and lakes for Sockeye Salmon) a multiplier of 1.5x the length of the stream segments flowing through the polygons was applied.

\begin{sphinxuseclass}{cell}
\begin{sphinxuseclass}{tag_remove-input}
\begin{sphinxuseclass}{tag_remove-output}
\end{sphinxuseclass}
\end{sphinxuseclass}
\end{sphinxuseclass}
\begin{figure}[htbp]
\centering
\capstart
\begin{sphinxVerbatimOutput}

\begin{sphinxuseclass}{cell_output}
\begin{sphinxVerbatim}[commandchars=\\\{\}]
\PYGZlt{}pandas.io.formats.style.Styler at 0x167dd29dc88\PYGZgt{}
\end{sphinxVerbatim}

\end{sphinxuseclass}\end{sphinxVerbatimOutput}
\caption{\sphinxstyleemphasis{Parameters and thresholds used to inform the Intrinsic Potential habitat model for spawning and rearing habitat for each target species in the Horsefly River watershed.}}\label{\detokenize{AppendixA:table15}}\end{figure}

\sphinxstepscope


\chapter{Appendix B}
\label{\detokenize{AppendixB:appendix-b}}\label{\detokenize{AppendixB::doc}}
\begin{sphinxuseclass}{cell}
\begin{sphinxuseclass}{tag_remove-input}
\begin{sphinxuseclass}{tag_remove-output}
\end{sphinxuseclass}
\end{sphinxuseclass}
\end{sphinxuseclass}

\section{Horsefly River Watershed Barrier Prioritization Summary}
\label{\detokenize{AppendixB:horsefly-river-watershed-barrier-prioritization-summary}}
\sphinxAtStartPar
The primary conservation outcome of the WCRP will be the remediation of barriers to connectivity in the Horsefly River watershed. To achieve Goal 1 in this plan, it is necessary to prioritize and identify a suite of barriers that, if remediated, will provide access to a minimum of \DUrole{pasted-text}{15.81} km of spawning or rearing habitat (\hyperref[\detokenize{AppendixB:table16}]{Fig.\@ \ref{\detokenize{AppendixB:table16}}}):

\begin{sphinxuseclass}{cell}
\begin{sphinxuseclass}{tag_remove-input}
\begin{sphinxuseclass}{tag_remove-output}
\end{sphinxuseclass}
\end{sphinxuseclass}
\end{sphinxuseclass}
\begin{figure}[htbp]
\centering
\capstart
\begin{sphinxVerbatimOutput}

\begin{sphinxuseclass}{cell_output}
\begin{sphinxVerbatim}[commandchars=\\\{\}]
\PYGZlt{}pandas.io.formats.style.Styler at 0x202bebc5448\PYGZgt{}
\end{sphinxVerbatim}

\end{sphinxuseclass}\end{sphinxVerbatimOutput}
\caption{\sphinxstyleemphasis{Spawning and rearing habitat connectivity gain requirements to meet WCRP goals in the Horsefly River watershed. The measures of currently accessible and total habitat values are derived from the Intrinsic Potential habitat model described in Appendix B.}}\label{\detokenize{AppendixB:table16}}\end{figure}

\sphinxAtStartPar
The barrier prioritization analysis ranked barriers by the amount of habitat blocked to produce an “intermediate barrier list” comprising more barriers than are needed to achieve the goals. A longer list of barriers is needed due to the inherent assumptions in the connectivity model, habitat model, and gaps in available data. Barriers that have been modelled (i.e., points where streams and road/rail networks intersect) are assumed to be barriers until field verification is undertaken and structures that have been assessed as “potential” barriers (e.g., may be passable at certain flow levels or for certain life history stages) require further investigation before a definitive remediation decision is made. Additionally, the habitat model identifies stream segments that have the potential to support spawning or rearing habitat for target species but does not attempt to quantify habitat quality or suitability (see Appendix B), which will require additional field verification once barrier assessments have completed. As such, the intermediate list of barriers below (\hyperref[\detokenize{AppendixB:table18}]{Fig.\@ \ref{\detokenize{AppendixB:table18}}}) should be considered as a starting point in the prioritization process and represents structures that are a priority to evaluate further through barrier assessment and habitat confirmations because some structures will likely be passable, others will not be associated with usable habitat, and others may not be feasible to remediate because of logistic considerations. The intermediate barrier list was updated following the barrier assessments and habitat confirmations that were undertaken during the 2021 field season \sphinxhyphen{} some barriers were moved forward to the “priority barrier list” (see \hyperref[\detokenize{AppendixB:table19}]{Fig.\@ \ref{\detokenize{AppendixB:table19}}}) and others were eliminated from consideration due to one or more of the considerations discussed above (see \hyperref[\detokenize{AppendixB:table17}]{Fig.\@ \ref{\detokenize{AppendixB:table17}}}). The priority barrier list represents structures that were confirmed to be partial or full barriers to fish passage and that block access to confirmed habitat. Barriers on the priority list were reviewed by planning team members and selected for inclusion for proactive pursual of remediation.  For more details on the barrier prioritization model, please see {[}\hyperlink{cite.references:id12}{Mazany\sphinxhyphen{}Wright \sphinxstyleemphasis{et al.}, 2021}{]}.

\begin{sphinxuseclass}{cell}
\begin{sphinxuseclass}{tag_remove-input}
\begin{sphinxuseclass}{tag_remove-output}
\end{sphinxuseclass}
\end{sphinxuseclass}
\end{sphinxuseclass}
\begin{figure}[htbp]
\centering
\capstart
\begin{sphinxVerbatimOutput}

\begin{sphinxuseclass}{cell_output}
\begin{sphinxVerbatim}[commandchars=\\\{\}]
\PYGZlt{}pandas.io.formats.style.Styler at 0x202c4e47ec8\PYGZgt{}
\end{sphinxVerbatim}

\end{sphinxuseclass}\end{sphinxVerbatimOutput}
\caption{\sphinxstyleemphasis{List of barriers that were prioritized as part of the first iteration of the intermediate barrier list (field assessments occurred during the 2021 field season) but were removed from consideration for pursual of proactive remediation following discussion with the planning team due to these structures not existing, being passable, not be associated with usable habitat, or deemed not feasible to remediate because of logistic considerations.}}\label{\detokenize{AppendixB:table17}}\end{figure}

\begin{sphinxuseclass}{cell}
\begin{sphinxuseclass}{tag_remove-input}
\begin{sphinxuseclass}{tag_full-width}
\begin{sphinxuseclass}{tag_remove-output}
\end{sphinxuseclass}
\end{sphinxuseclass}
\end{sphinxuseclass}
\end{sphinxuseclass}

\begin{wrapfigure}{l}{1000\sphinxpxdimen}
\centering\begin{sphinxVerbatimOutput}

\begin{sphinxuseclass}{cell_output}
\begin{sphinxVerbatim}[commandchars=\\\{\}]
\PYGZlt{}pandas.io.formats.style.Styler at 0x202beabad08\PYGZgt{}
\end{sphinxVerbatim}

\end{sphinxuseclass}\end{sphinxVerbatimOutput}
\caption{\sphinxstyleemphasis{Updated intermediate barrier list resulting from the second barrier prioritization analysis in the Horsefly River watershed. After assessing the potential barriers on the first iteration of the intermediate list (2021 field season) and either identifying them as remediation priorities (see \hyperref[\detokenize{AppendixB:table18}]{Fig.\@ \ref{\detokenize{AppendixB:table18}}}) or eliminating them from consideration (e.g., because they passed fish or did hot have suitable habitat upstream), the remaining potential barriers in the watershed were re\sphinxhyphen{}prioritized. The barriers on this list were prioritized to exceed the connectivity goals of the plan. Barriers highlighted in the same colour represent sets of barriers that have been prioritized as a group. In the Barrier Status column, P = potential barrier and B = confirmed barrier. All barrier assessment data is compiled from the BC Provincial Stream Crossing Inventory System.}}\label{\detokenize{AppendixB:table18}}\end{wrapfigure}

\begin{sphinxuseclass}{cell}
\begin{sphinxuseclass}{tag_remove-input}
\begin{sphinxuseclass}{tag_full-width}
\begin{sphinxuseclass}{tag_remove-output}
\end{sphinxuseclass}
\end{sphinxuseclass}
\end{sphinxuseclass}
\end{sphinxuseclass}

\begin{wrapfigure}{l}{1100\sphinxpxdimen}
\centering\begin{sphinxVerbatimOutput}

\begin{sphinxuseclass}{cell_output}
\begin{sphinxVerbatim}[commandchars=\\\{\}]
\PYGZlt{}pandas.io.formats.style.Styler at 0x202c4f6dcc8\PYGZgt{}
\end{sphinxVerbatim}

\end{sphinxuseclass}\end{sphinxVerbatimOutput}
\caption{\sphinxstyleemphasis{The Horsefly River watershed priority barrier list, which includes barriers that have undergone field assessment, been reviewed by the planning team, and selected to pursue for proactive remediation.}}\label{\detokenize{AppendixB:table19}}\end{wrapfigure}

\sphinxAtStartPar
Out of the \DUrole{pasted-text}{13} on the intermediate list, 16 require further field assessment before selection as a final barrier to pursue for remediation:

\begin{sphinxuseclass}{cell}
\begin{sphinxuseclass}{tag_remove-input}
\begin{sphinxuseclass}{tag_remove-output}
\end{sphinxuseclass}
\end{sphinxuseclass}
\end{sphinxuseclass}
\begin{figure}[htbp]
\centering
\capstart
\begin{sphinxVerbatimOutput}

\begin{sphinxuseclass}{cell_output}
\begin{sphinxVerbatim}[commandchars=\\\{\}]
\PYGZlt{}pandas.io.formats.style.Styler at 0x202c4edf3c8\PYGZgt{}
\end{sphinxVerbatim}

\end{sphinxuseclass}\end{sphinxVerbatimOutput}
\caption{\sphinxstyleemphasis{Field assessment requirements for the intermediate barrier list in the Horsefly River watershed. The cost per barrier values are estimates based on previously completed field work. The habitat confirmation count is based on the assumption that the 12 barriers requiring barrier assessments will also require a subsequent confirmation. In the case that some barriers are identified as unsuitable candidates for habitat confirmations, the total cost will be reduced.}}\label{\detokenize{AppendixB:table20}}\end{figure}

\sphinxAtStartPar
There are currently \DUrole{pasted-text}{13} barriers on the priority barrier list, which will be pursued for proactive remediation to achieve the connectivity goals in this plan:

\begin{sphinxuseclass}{cell}
\begin{sphinxuseclass}{tag_remove-input}
\begin{sphinxuseclass}{tag_remove-output}
\end{sphinxuseclass}
\end{sphinxuseclass}
\end{sphinxuseclass}
\begin{figure}[htbp]
\centering
\capstart
\begin{sphinxVerbatimOutput}

\begin{sphinxuseclass}{cell_output}
\begin{sphinxVerbatim}[commandchars=\\\{\}]
\PYGZlt{}pandas.io.formats.style.Styler at 0x202c4f8f3c8\PYGZgt{}
\end{sphinxVerbatim}

\end{sphinxuseclass}\end{sphinxVerbatimOutput}
\caption{\sphinxstyleemphasis{Preliminary barrier remediation cost estimate to reach connectivity goals in the Horsefly River watershed. Cost per barrier values are estimated based on the average cost of previously completed projects. Barrier counts and total costs are subject to change as more information is collected through the implementation of this plan.}}\label{\detokenize{AppendixB:table21}}\end{figure}

\begin{sphinxthebibliography}{tQelmucw}
\bibitem[ASB+05]{references:id2}
\sphinxAtStartPar
A Agrawal, R S Schick, E P Bjorkstedt, R G Szerlong, M N Goslin, B C Spence, T H Williams, and K M Burnett. Predicting the potential for historical coho, chinook, and steelhead habitat in northern california. \sphinxstyleemphasis{National Oceanic and Atmospheric Administration}, 2005.
\bibitem[BR91]{references:id3}
\sphinxAtStartPar
T C Bjornn and D W Reiser. Habitat requirements of salmonids in streams. \sphinxstyleemphasis{Influences of Forest and Rangeland Management on Salmonid Fishes and their Habitats}, 19:83–138, 1991.
\bibitem[BRM+07]{references:id4}
\sphinxAtStartPar
Kelly M Burnett, Gordon H Reeves, Daniel J Miller, Sharon Clarke, Ken Vance\sphinxhyphen{}Borland, and Kelly Christiansen. Distribution of salmon\sphinxhyphen{}habitat potential relative to landscape characteristics and implications for conservation. \sphinxstyleemphasis{Ecol. Appl.}, 17(1):66–80, January 2007.
\bibitem[BSB+11]{references:id5}
\sphinxAtStartPar
D S Busch, M Sheer, K Burnett, P Mcelhany, and T Cooney. Landscape\sphinxhyphen{}level model to predict spawning habitat for lower columbia river fall chinook salmon (oncorhynchus tshawytscha). \sphinxstyleemphasis{River Research Applications}, 29:291–312, 2011.
\bibitem[CH06]{references:id6}
\sphinxAtStartPar
T Cooney and D Holzer. \sphinxstyleemphasis{Appendix C: Interior Columbia basin stream type Chinook Salmon and Steelhead populations: habitat intrinsic potential analysis. National Oceanic and Atmospheric Administration}. Northwest Fisheries Center, Northwest Fisheries Center, 2006.
\bibitem[COS16]{references:id7}
\sphinxAtStartPar
COSEWIC. \sphinxstyleemphasis{COSEWIC Assessment and Status Report on the Coho Salmon Oncorhynchus kisutch, Interior Fraser Population, in Canada. Committee on the Status of Endangered Wildlife in Canada, Ottawa, Ontario.} ECC, https://www.canada.ca/en/environment\sphinxhyphen{}climate\sphinxhyphen{}change/services/species\sphinxhyphen{}risk\sphinxhyphen{}public\sphinxhyphen{}registry/cosewic\sphinxhyphen{}assessments\sphinxhyphen{}status\sphinxhyphen{}reports/coho\sphinxhyphen{}salmon\sphinxhyphen{}interior\sphinxhyphen{}fraser\sphinxhyphen{}2016.html., 2016.
\bibitem[COS17]{references:id8}
\sphinxAtStartPar
COSEWIC. \sphinxstyleemphasis{COSEWIC Assessment and Status Report on the Sockeye Salmon Oncorhynchus nerka, 24 Designatable Units in the Fraser River Drainage Basin, in Canada. Committee on the Status of Endangered Wildlife in Canada, Ottawa, Ontario.} ECC, https://www.sararegistry.gc.ca/virtual\_sara/files/cosewic/srSockeyeSalmon2017e.pdf., 2017.
\bibitem[COS18]{references:id9}
\sphinxAtStartPar
COSEWIC. \sphinxstyleemphasis{COSEWIC Assessment and Status Report on the Chinook Salmon Oncorhynchus tshawytscha, Designatable Units in Southern British Columbia (Part One – Designatable Units with no or low levels of artificial releases in the last 12 years), in Canada. Committee on the Status of Endangered Wildlife in Canada, Ottawa, Ontario.} ECC, https://www.canada.ca/en/environment\sphinxhyphen{}climate\sphinxhyphen{}change/services/species\sphinxhyphen{}risk\sphinxhyphen{}public\sphinxhyphen{}registry/cosewic\sphinxhyphen{}assessments\sphinxhyphen{}status\sphinxhyphen{}reports/chinook\sphinxhyphen{}salmon\sphinxhyphen{}2018/document\sphinxhyphen{}info\sphinxhyphen{}summaries.html., 2018.
\bibitem[DFO91]{references:id11}
\sphinxAtStartPar
DFO. \sphinxstyleemphasis{Fish habitat inventory and information program \sphinxhyphen{} stream summary information.} DFO, 1991.
\bibitem[Lak99]{references:id10}
\sphinxAtStartPar
R G Lake. \sphinxstyleemphasis{Activity and spawning behaviour in spawning Sockeye salmon. Thesis}. UBC, 1999.
\bibitem[Ltd18]{references:id17}
\sphinxAtStartPar
Masse Environmental Consultants Ltd. \sphinxstyleemphasis{Fish Habitat Confirmation Assessments Horsefly River Watershed. Prepared for Ministry of Environment \& Climate Change Strategy.} Masse Environmental Consultants Ltd., 2018.
\bibitem[MWNLR21a]{references:id12}
\sphinxAtStartPar
N Mazany\sphinxhyphen{}Wright, S M Norris, N W R Lapointe, and B Rebellato. A freshwater connectivity modelling framework to support barrier prioritization and remediation in british columbia. \sphinxstyleemphasis{Canadian Wildlife Federation}, 2021.
\bibitem[MWNLR21b]{references:id13}
\sphinxAtStartPar
N Mazany\sphinxhyphen{}Wright, S M Norris, N W R Lapointe, and B Rebellato. Fish passage restoration initiative target watershed selection process: technical documentation. \sphinxstyleemphasis{Canadian Wildlife Federation}, 2021.
\bibitem[MWNS+21]{references:id14}
\sphinxAtStartPar
N Mazany\sphinxhyphen{}Wright, J Noseworthy, S Sra, S M Norris, and N W Lapointe. Breaking down barriers: a practitioners' guide to watershed connectivity remediation planning. \sphinxstyleemphasis{Canadian Wildlife Federation}, 2021.
\bibitem[Mcm83]{references:id15}
\sphinxAtStartPar
T E Mcmahon. Habitat suitability index models: coho salmon. \sphinxstyleemphasis{U.S. Department of the Interior, Fish and Wildlife Service}, 1983.
\bibitem[Nation21a]{references:id28}
\sphinxAtStartPar
Williams Lake First Nation. \sphinxstyleemphasis{Secwepemc Land Use Patterns.} https://www.wlfn.ca/about\sphinxhyphen{}wlfn/history/., 2021.
\bibitem[Nation21b]{references:id31}
\sphinxAtStartPar
Xatśūll First Nation. Traditional history. \sphinxstyleemphasis{XFN}, 2021.
\bibitem[NN77]{references:id16}
\sphinxAtStartPar
H R Neuman and C P Newcombe. \sphinxstyleemphasis{Minimum acceptable stream flows in British Columbia: a review}. Fisheries Management Report No. 70., 1977.
\bibitem[PSF20]{references:id19}
\sphinxAtStartPar
Pacific\sphinxhyphen{}Salmon\sphinxhyphen{}Foundation. \sphinxstyleemphasis{Methods for Assessing Status and Trends in Pacific Salmon Conservation Units and their Freshwater Habitats. The Pacific Salmon Foundation, Vancouver, British Columbia.} PSF, 2020.
\bibitem[PPWB08]{references:id20}
\sphinxAtStartPar
M Porter, D Pickard, K Wieckowski, and K Bryan. \sphinxstyleemphasis{Developing Fish Habitat Models for Broad\sphinxhyphen{}Scale Forest Planning in the Southern Interior of B.C.} ESSA Technologies Ltd. and B.C. Ministry of Environment, 2008.
\bibitem[RM86]{references:id21}
\sphinxAtStartPar
R F Raleigh and W J Miller. \sphinxstyleemphasis{Habitat suitability index models and instream flow suitability curves: chinook salmon. U.S. Fish and Wildlife Service Biological Reports 82}. USFW, 1986.
\bibitem[RHMS02]{references:id22}
\sphinxAtStartPar
M Roberge, J B M Hume, C K Minns, and T Slaney. \sphinxstyleemphasis{Life history characteristics of freshwater fishes occurring in British Columbia and the Yukon, with major emphasis on stream habitat characteristics.} Fisheries and Oceans Canada, Marine Environment and Habitat Science Division, Cultus Lake, British Columbia, 2002.
\bibitem[RPP00]{references:id23}
\sphinxAtStartPar
Jordan Rosenfeld, Marc Porter, and Eric Parkinson. Habitat factors affecting the abundance and distribution of juvenile cutthroat trout (oncorhynchus clarki) and coho salmon (oncorhynchus kisutch). \sphinxstyleemphasis{Can. J. Fish. Aquat. Sci.}, 57(4):766–774, April 2000.
\bibitem[SZ18]{references:id24}
\sphinxAtStartPar
Carina Seliger and Bernhard Zeiringer. River connectivity, habitat fragmentation and related restoration measures. In \sphinxstyleemphasis{Riverine Ecosystem Management}, pages 171–186. Springer International Publishing, Cham, 2018.
\bibitem[SBG+09]{references:id25}
\sphinxAtStartPar
M B Sheer, D S Busch, E Gilbert, J M Bayer, S Lanigan, J L Schei, K M Burnett, and D Miller. \sphinxstyleemphasis{Development and management of fish intrinsic potential data and methodologies: State of the IP 2008 summary report}. Pacific Northwest Aquatic Monitoring Partnership Series 2009—4, 56 pp., 2009.
\bibitem[SS06]{references:id26}
\sphinxAtStartPar
M B Sheer and E A Steel. Lost watersheds: barriers, aquatic habitat connectivity, and salmon persistence in the willamette and lower columbia river basins. \sphinxstyleemphasis{Trans. Am. Fish. Soc.}, 135(6):1654–1669, November 2006.
\bibitem[SRC17]{references:id27}
\sphinxAtStartPar
Matthew R Sloat, Gordon H Reeves, and Kelly R Christiansen. Stream network geomorphology mediates predicted vulnerability of anadromous fish habitat to hydrologic change in southeast alaska. \sphinxstyleemphasis{Glob. Chang. Biol.}, 23(2):604–620, February 2017.
\bibitem[tQelmucw19]{references:id18}
\sphinxAtStartPar
Northern Secwepemc to Qelmucw. \sphinxstyleemphasis{Saving the Salmon. Lexéy’em – Fall 2019.} https://nstq.ca/wp\sphinxhyphen{}content/uploads/2019/09/Lexeyem\_Fall\sphinxhyphen{}2019\_Final.pdf., 2019.
\bibitem[WTD98]{references:id29}
\sphinxAtStartPar
I. R. Wilson, K. Twohig, and B. Dahlstrom. \sphinxstyleemphasis{Archaeological Overview Assessment Northern Secwepemc Traditional Territory.} https://www2.gov.bc.ca/assets/gov/farming\sphinxhyphen{}natural\sphinxhyphen{}resources\sphinxhyphen{}and\sphinxhyphen{}industry/natural\sphinxhyphen{}resource\sphinxhyphen{}use/archaeology/forms\sphinxhyphen{}publications/aoa\_\sphinxhyphen{}\_williams\_lake\_\sphinxhyphen{}\_northern\_secwepemc\_traditional\_territory\_\sphinxhyphen{}\_1998\_report.pdf., 1998.
\bibitem[WAW17]{references:id30}
\sphinxAtStartPar
C Woll, D Albert, and D Whited. \sphinxstyleemphasis{A Preliminary Classification and Mapping of Salmon Ecological Systems in the Nushagak and Kvichak Watersheds}. The Nature Conservancy, Alaska, 2017.
\end{sphinxthebibliography}







\renewcommand{\indexname}{Index}
\printindex
\end{document}